\documentclass[11pt, oneside]{article}
\usepackage[T1]{fontenc}

\usepackage[letterpaper, margin=1in]{geometry}

\usepackage{booktabs}
\usepackage{listings}
\usepackage{xcolor}

\usepackage{amsmath}
\usepackage{amsthm}
\usepackage{amssymb}

\usepackage{hyperref}

\bibliographystyle{plain}

\title{POLS 4724: Practicum Pre-Analysis Plan}
\author{Aman Choudhri}
\date{\today}

\newcommand{\dataset}[2]{
    \href{#2}{\texttt{#1}}
}

\newcommand{\ImageNet}{\dataset{ImageNet}{https://www.image-net.org/about.php}}

\lstdefinestyle{R}{
    language=R,
    basicstyle=\small\ttfamily,
    commentstyle=\color{green!60!black},
    keywordstyle=\color{blue},
    stringstyle=\color{red},
    breaklines=true,
    breakatwhitespace=true,
    showstringspaces=false,
    numbers=left,
    numberstyle=\tiny\color{gray},
    frame=single,
    rulecolor=\color{black},
}
\renewcommand{\lstlistingname}{Code}

\begin{document}
\maketitle

\section*{Motivation}

Comparing the performance of different machine learning (ML) algorithms in practice
is made difficult by the cost of evaluating a given model. A single training
run, even on a GPU in an academic compute cluster, may take upwards of 2 days
for common image classification benchmark datasets like \ImageNet. These compute
requirements naturally limit the extent to which budget-constrained researchers
can systematically compare proposed novel techniques to state-of-the-art
baselines. Rather than running a proposed algorithm head-to-head with a
baseline on a variety of benchmark datasets, researchers may be forced to limit
the scope of their comparisons to just a few hand-picked datasets—which they
hope will be illustrative of algorithm performance as a whole.

This may be viewed as a problem of ``superpopulation inference.'' Formally,
researchers observe some performance difference between two algorithms on some
subset $d \ll D$ of some available set of $D$ benchmarks, and hope that their
estimates satisfy $\hat{\tau}_d \approx \hat{\tau}_D$. Of course, this poses some problems.
The subset $d$ may not be ``representative'' of the broader set of benchmarks $D$, by which
I mean that observed performance improvement from algorithms on $d$ may fail to hold
true on $D$ in general. This issue makes it hard to declare categorically that any
one algorithm is better than another.

% Complicating this problem is an issue of \emph{hyperparameters}, tunable configuration
% options that a researcher can specify to modify the behavior of any given algorithm. It's
% often found in practice that identifying an optimal hyperparameter configuration
% is crucial

In this practicum, I propose a somewhat silly idea to alleviate the
computational burden of systematic performance comparisons: randomized
experimentation. I describe this idea as silly, because in this case we may
reasonably be able to observe both potential outcomes. We might run two
different ML algorithms on the same dataset, obtain performance measurements
$Y_i^{(0)}, Y_i^{(1)}$, and compare them as if they were potential outcomes to
observe a true ``individual treatment effect.'' But I maintain that it is
nonetheless an interesting question to study how we should best allocate a
fixed evaluation budget across models and benchmark datasets, assuming we
cannot run all our models of interest on all our datasets. 

Specifically, my question of interest is the following: can a randomized
experiment in which $D$ benchmark datasets are randomly assigned to two
different ML algorithms, $z \in \left\{ 0, 1 \right\}$, recover the true
performance difference between $z = 0$ and $z = 1$ better than an exact
performance difference observation on a handpicked finite subset $d$ of the
benchmarks?

\section*{Plan}

The subjects of this experiment will be $D = 30$ common supervised image
classification datasets. Our outcome of interest is the test-set classification
accuracy of an ML model trained on the dataset, and the choice of model will
represent our treatment/control conditions.

\subsection*{Subjects}
I selected these datasets from a collection of open-access datasets made available
through the ML Python package \texttt{torchvision}. I restricted to datasets
whose task was image classification, meaning a ML model should input an image
and output an integer representing the ``class'' to which the image should
belong. I also removed from consideration datasets that would take too long to
train on, like the aforementioned \ImageNet dataset.

For each dataset, we have the following covariates: number of images, number of
classes, and average image size. All are measured pre-treatment. Since images
are often of different shapes within a dataset but standardized sizes for each
dataset is a requirement for modern deep learning algorithms, we standardize
sizes as follows: We'll compute the average size of the images within a datset,
round to the nearest multiple of 4, and resize all images within a dataset to
that size. For larger images with more than 10,000 pixels, we round to the
nearest multiple of 32. We'll define the computed covariate $S_i$ as the
number of pixels in the standardized images for each dataset.


\subsection*{Treatment}
The control condition in this experiment is the convolutional neural network
(CNN) architecture, which was introduced in 1998
\cite{lecunGradientbasedLearningApplied1998} and popularized in 2012
\cite{krizhevskyImageNetClassificationDeep2012}. It remains one of the most
popular neural network setups for image processing to this day. We'll denote it
by $z = 0$. The treatment condition, written as $z = 1$, is the more novel
vision transformer (ViT) architecture \cite{dosovitskiyImageWorth16x162021},
introduced in 2020.

\subsection*{Outcome}
For each dataset $i$, our outcomes of interest are $Y_i(0)$ and $Y_i(1)$, the
classification accuracies of the assigned model on an unseen test set.
Specifically, a given model will be trained only on a \emph{subset} of a
dataset $i$, with the rest of the data set aside and dubbed the ``test set.''
After training, then, the model will produce predicted class assignments for
each unseen image in the test set. The classification accuracy is simply the
proportion of correct class assignments by the model. Since the outcome is a proportion,
we note that $0 \leq Y_i(z) \leq 1$.

\subsection*{Randomization}
Since we have access to covariates from the datasets, we will use blocked
random assignment. One covariate in particular is relevant here, what I'll call
the \emph{richness}: the average number of images per class. Roughly speaking,
the more images there are available for each class, the easier the task since
there is more information available for the model to pull from. I'll block on
richness thresholding at 100 examples per class, since there are comparatively
few datasets in that regime and I want to make sure we try both models on those
kinds of datasets. This results in the following group sizes:

\begin{table}[htbp]
\centering
\caption{Number of Datasets by Richness}
\label{tab:richness_blocking}
\begin{tabular}{llr}
\toprule
 & Richness Range & Count \\
$R_{i}$ &  &  \\
\midrule
0 & < 100 examples/class & 7 \\
1 & > 100 examples/class & 23 \\
\bottomrule
\end{tabular}
\end{table}


We'll use complete random assignment within each block. For $R_i = 0$, we'll
assign 3 datasets to treatment and 4 to control. For $R_i = 1$, we'll assign 11
to treatment and 12 to control.

The silly nature of the randomization here means we won't face noncompliance
and that we are safe to assume excludability and non-interference. However, we
may face an issue of attrition, with models failing to train in time or failing
to converge at all. This may correlate with potential outcomes, as datasets
with smaller potential classification accuracies $Y_i(z)$ may be harder to train on,
meaning we have a higher chance of attrition among those datasets. We discuss
how to respond to this in the following section.

\subsection*{Analysis}
For this experiment, I'm primarily interested in the average treatment effect
\[
    \mathbb{E}[Y_i(1) - Y_i(0)]
,\] 
which I'll estimate using the block-adjusted difference-in-means estimator.
However, we also have access to the \emph{log image size} covariate, $S_i$.
Generally the larger the image, the harder the classification task—so I'd like
to adjust for $S_i$ as well.

To compute an ATE estimate using both a blocking indicator $R_i$ and a covariate $S_i$ , we
will use weighted regression of the form
\[
    Y_i \sim \alpha + \beta z_i + \gamma \log(S_i)
.\] 
We transform the image size $S_i$ into logarithm scale before the regression,
as a one-unit difference on the log scale is more meaningful and consistent
across a variety of scales.

We will need to use a weighted regression since the treatment assignment
probabilities per block are not exactly the same. The weights for this
regression are given by equation 4.12 in
\cite{gerberFieldExperimentsDesign2012}: the weight for subject $i$ in block
$j$ is given by
\[
    w_{ij} = \frac{1}{p_{ij}}z_i + (\frac{1}{1- p_{ij}})(1 - z_i)
,\] 
where $p_{ij}$ is the probability that subject $i$ within block $j$ was
assigned to treatment.

I hypothesize no treatment effect. I'll test this hypothesis using
the linear regression analysis, using a two-tailed test on the coefficient
representing our estimate $\hat{\beta}$.

As discussed previously, we face a potential challenge of attrition here, with
missing data likely being correlated with potential outcomes $Y_i(z)$. If
attrition occurs, we will conduct the above analysis among the results that we
do have, the Always-Reporters. We won't attempt any of the correction methods,
since the introduction of randomness may already add too much noise to discern
any true performance differences. If attrition does occur, we leave it to a
follow-up study to retry the outcome measurement on the subjects, relying on
the nature of our problem to assume that these will return the same potential
outcomes.

\newpage

\bibliography{references}
\newpage

\section*{Appendix}

\subsection*{Datasets}


\begin{table}[htbp]
\centering
\caption{Overview of Image Classification Datasets}
\label{tab:datasets}
\begin{tabular}{llrrr}
\toprule
 & Task & Size & Num Classes & Standardized Image Size \\
\midrule
\href{https://www.cs.toronto.edu/~kriz/cifar.html}{\texttt{CIFAR10}} & General objects & 60,000 & 10 & 32x32 \\
\href{https://www.cs.toronto.edu/~kriz/cifar.html}{\texttt{CIFAR100}} & General objects & 60,000 & 100 & 32x32 \\
\href{https://data.caltech.edu/records/mzrjq-6wc02}{\texttt{Caltech101}} & General objects & 9,146 & 101 & 320x224 \\
\href{https://data.caltech.edu/records/20087}{\texttt{Caltech256}} & General objects & 30,607 & 256 & 320x224 \\
\href{https://mmlab.ie.cuhk.edu.hk/projects/CelebA.html}{\texttt{CelebA}} & Face & 202,599 & 10177 & 192x224 \\
\href{https://github.com/openai/CLIP/blob/main/data/country211.md}{\texttt{Country211}} & Countries & 75,000 & 249 & 480x384 \\
\href{https://www.robots.ox.ac.uk/~vgg/data/dtd/}{\texttt{DTD}} & Textures & 5,640 & 47 & 512x480 \\
\href{https://www.westernsydney.edu.au/icns/resources/reproducible_research3/publication_support_materials2/emnist}{\texttt{EMNIST}} & Handwriting & 814,255 & 47 & 28x28 \\
\href{https://github.com/phelber/eurosat}{\texttt{EuroSAT}} & Land & 27,000 & 10 & 64x64 \\
\href{https://www.kaggle.com/c/challenges-in-representation-learning-facial-expression-recognition-challenge}{\texttt{FER2013}} & Faces & 35,887 & 7 & 48x48 \\
\href{https://www.robots.ox.ac.uk/~vgg/data/fgvc-aircraft/}{\texttt{FGVCAircraft}} & Aircraft & 10,000 & 100 & 1120x768 \\
\href{https://github.com/zalandoresearch/fashion-mnist}{\texttt{FashionMNIST}} & Clothing & 70,000 & 10 & 28x28 \\
\href{https://www.robots.ox.ac.uk/~vgg/data/flowers/102/}{\texttt{Flowers102}} & Flowers & 7,169 & 102 & 640x544 \\
\href{https://data.vision.ee.ethz.ch/cvl/datasets_extra/food-101/}{\texttt{Food101}} & Food & 101,000 & 101 & 512x512 \\
\href{https://benchmark.ini.rub.de/}{\texttt{GTSRB}} & Traffic Signs & 51,840 & 43 & 52x52 \\
\href{https://github.com/fastai/imagenette#imagenette-1}{\texttt{Imagenette}} & General objects & 13,394 & 10 & 480x416 \\
\href{https://github.com/rois-codh/kmnist}{\texttt{KMNIST}} & Handwriting & 70,000 & 10 & 28x28 \\
\href{http://vis-www.cs.umass.edu/lfw/}{\texttt{LFWPeople}} & Faces & 13,233 & 5749 & 256x256 \\
\href{http://yann.lecun.com/exdb/mnist/}{\texttt{MNIST}} & Handwriting & 70,000 & 10 & 28x28 \\
\href{https://github.com/brendenlake/omniglot}{\texttt{Omniglot}} & Handwriting & 32,460 & 964 & 128x128 \\
\href{https://www.robots.ox.ac.uk/~vgg/data/pets/}{\texttt{OxfordIIITPet}} & Animals & 7,400 & 37 & 448x416 \\
\href{https://github.com/basveeling/pcam}{\texttt{PCAM}} & Medical scans & 327,680 & 2 & 96x96 \\
\href{https://github.com/facebookresearch/qmnist}{\texttt{QMNIST}} & Handwriting & 120,000 & 10 & 28x28 \\
\href{https://github.com/openai/CLIP/blob/main/data/rendered-sst2.md}{\texttt{RenderedSST2}} & Sentiment from text images & 8,741 & 2 & 448x448 \\
\href{https://archive.ics.uci.edu/dataset/178/semeion+handwritten+digit}{\texttt{SEMEION}} & Handwriting & 1,593 & 10 & 16x16 \\
\href{https://cs.stanford.edu/~acoates/stl10/}{\texttt{STL10}} & General objects & 13,000 & 10 & 96x96 \\
\href{https://vision.princeton.edu/projects/2010/SUN/}{\texttt{SUN397}} & Scenes & 108,753 & 397 & 960x800 \\
\href{http://ufldl.stanford.edu/housenumbers/}{\texttt{SVHN}} & Digits & 630,420 & 10 & 32x32 \\
\href{https://www.kaggle.com/datasets/jessicali9530/stanford-cars-dataset}{\texttt{StanfordCars}} & Cars & 16,185 & 196 & 704x512 \\
\href{https://www.csie.ntu.edu.tw/~cjlin/libsvmtools/datasets/multiclass.html#usps}{\texttt{USPS}} & Handwriting & 9,298 & 10 & 16x16 \\
\bottomrule
\end{tabular}
\end{table}


\newpage

\section*{Code}

First, here is the code to generate random assignments.

\lstinputlisting[
    language=R,
    style=R,
    caption={Randomization code, from \texttt{randomization.R}},
    label={lst:randomization}
]{../analysis/randomization.R}

\newpage

Here is the code for the regression analysis.

\lstinputlisting[
    language=R,
    style=R,
    caption={Analysis code, from \texttt{main.R}},
    label={lst:analysis}
]{../analysis/main.R}

\end{document}
