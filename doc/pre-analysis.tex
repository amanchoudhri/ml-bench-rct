\documentclass[11pt, oneside]{article}
\usepackage[T1]{fontenc}

\usepackage[letterpaper, margin=1in]{geometry}

\usepackage{booktabs}

\usepackage{hyperref}

\title{POLS 4724: Practicum Pre-Analysis Plan}
\author{Aman Choudhri}
\date{\today}

\newcommand{\dataset}[2]{
    \href{#2}{\texttt{#1}}
}

\newcommand{\ImageNet}{\dataset{ImageNet}{https://www.image-net.org/about.php}}

\begin{document}
\maketitle

\section*{Motivation}

Comparing the performance of different machine learning (ML) algorithms in practice
is made difficult by the cost of evaluating a given model. A single training
run, even on a GPU in an academic compute cluster, may take upwards of 2 days
for common image classification benchmark datasets like \ImageNet. These compute
requirements naturally limit the extent to which budget-constrained researchers
can systematically compare proposed novel techniques to state-of-the-art
baselines. Rather than running a proposed algorithm head-to-head with a
baseline on a variety of benchmark datasets, researchers may be forced to limit
the scope of their comparisons to just a few hand-picked datasets—which they
hope will be illustrative of algorithm performance as a whole.

This may be viewed as a problem of ``superpopulation inference.'' Formally,
researchers observe some performance difference between two algorithms on some
subset $d \ll D$ of some available set of $D$ benchmarks, and hope that their
estimates satisfy $\hat{\tau}_d \approx \hat{\tau}_D$. Of course, this poses some problems.
The subset $d$ may not be ``representative'' of the broader set of benchmarks $D$, by which
I mean that observed performance improvement from algorithms on $d$ may fail to hold
true on $D$ in general. This issue makes it hard to declare categorically that any
one algorithm is better than another.

% Complicating this problem is an issue of \emph{hyperparameters}, tunable configuration
% options that a researcher can specify to modify the behavior of any given algorithm. It's
% often found in practice that identifying an optimal hyperparameter configuration
% is crucial

In this practicum, I propose a somewhat silly idea to alleviate the
computational burden of systematic performance comparisons: randomized
experimentation. I describe this idea as silly, because in this case we may
reasonably be able to observe both potential outcomes. We might run two
different ML algorithms on the same dataset, obtain performance measurements
$Y_i^{(0)}, Y_i^{(1)}$, and compare them as if they were potential outcomes to
observe a true ``individual treatment effect.'' But I maintain that it is
nonetheless an interesting question to study how we should best allocate a
fixed evaluation budget across models and benchmark datasets, assuming we
cannot run all our models of interest on all our datasets. 

Specifically, my question of interest is the following: can a randomized
experiment in which $D$ benchmark datasets are randomly assigned to two
different ML algorithms, $z \in \left\{ 0, 1 \right\}$, recover the true
performance difference between $z = 0$ and $z = 1$ better than an exact
performance difference observation on a handpicked finite subset $d$ of the
benchmarks?

\section*{Plan}

The subjects of this experiment will be $D = 30$ common supervised image classification
datasets.

I selected these from a collection of open-access datasets made available
through the ML Python package \texttt{torchvision}. I restricted to datasets
whose task was image classification, meaning a ML model should input an image
and output an integer representing the ``class'' to which the image should
belong. I also removed from consideration datasets that would take too long to
train on, like the aforementioned \ImageNet dataset.


\newpage

\section*{Appendix}

\subsection*{Datasets}


\begin{table}[htbp]
\centering
\caption{Overview of Image Classification Datasets}
\label{tab:datasets}
\begin{tabular}{llrrr}
\toprule
 & Task & Size & Num Classes & Standardized Image Size \\
\midrule
\href{https://www.cs.toronto.edu/~kriz/cifar.html}{\texttt{CIFAR10}} & General objects & 60,000 & 10 & 32x32 \\
\href{https://www.cs.toronto.edu/~kriz/cifar.html}{\texttt{CIFAR100}} & General objects & 60,000 & 100 & 32x32 \\
\href{https://data.caltech.edu/records/mzrjq-6wc02}{\texttt{Caltech101}} & General objects & 9,146 & 101 & 320x224 \\
\href{https://data.caltech.edu/records/20087}{\texttt{Caltech256}} & General objects & 30,607 & 256 & 320x224 \\
\href{https://mmlab.ie.cuhk.edu.hk/projects/CelebA.html}{\texttt{CelebA}} & Face & 202,599 & 10177 & 192x224 \\
\href{https://github.com/openai/CLIP/blob/main/data/country211.md}{\texttt{Country211}} & Countries & 75,000 & 249 & 480x384 \\
\href{https://www.robots.ox.ac.uk/~vgg/data/dtd/}{\texttt{DTD}} & Textures & 5,640 & 47 & 512x480 \\
\href{https://www.westernsydney.edu.au/icns/resources/reproducible_research3/publication_support_materials2/emnist}{\texttt{EMNIST}} & Handwriting & 814,255 & 47 & 28x28 \\
\href{https://github.com/phelber/eurosat}{\texttt{EuroSAT}} & Land & 27,000 & 10 & 64x64 \\
\href{https://www.kaggle.com/c/challenges-in-representation-learning-facial-expression-recognition-challenge}{\texttt{FER2013}} & Faces & 35,887 & 7 & 48x48 \\
\href{https://www.robots.ox.ac.uk/~vgg/data/fgvc-aircraft/}{\texttt{FGVCAircraft}} & Aircraft & 10,000 & 100 & 1120x768 \\
\href{https://github.com/zalandoresearch/fashion-mnist}{\texttt{FashionMNIST}} & Clothing & 70,000 & 10 & 28x28 \\
\href{https://www.robots.ox.ac.uk/~vgg/data/flowers/102/}{\texttt{Flowers102}} & Flowers & 7,169 & 102 & 640x544 \\
\href{https://data.vision.ee.ethz.ch/cvl/datasets_extra/food-101/}{\texttt{Food101}} & Food & 101,000 & 101 & 512x512 \\
\href{https://benchmark.ini.rub.de/}{\texttt{GTSRB}} & Traffic Signs & 51,840 & 43 & 52x52 \\
\href{https://github.com/fastai/imagenette#imagenette-1}{\texttt{Imagenette}} & General objects & 13,394 & 10 & 480x416 \\
\href{https://github.com/rois-codh/kmnist}{\texttt{KMNIST}} & Handwriting & 70,000 & 10 & 28x28 \\
\href{http://vis-www.cs.umass.edu/lfw/}{\texttt{LFWPeople}} & Faces & 13,233 & 5749 & 256x256 \\
\href{http://yann.lecun.com/exdb/mnist/}{\texttt{MNIST}} & Handwriting & 70,000 & 10 & 28x28 \\
\href{https://github.com/brendenlake/omniglot}{\texttt{Omniglot}} & Handwriting & 32,460 & 964 & 128x128 \\
\href{https://www.robots.ox.ac.uk/~vgg/data/pets/}{\texttt{OxfordIIITPet}} & Animals & 7,400 & 37 & 448x416 \\
\href{https://github.com/basveeling/pcam}{\texttt{PCAM}} & Medical scans & 327,680 & 2 & 96x96 \\
\href{https://github.com/facebookresearch/qmnist}{\texttt{QMNIST}} & Handwriting & 120,000 & 10 & 28x28 \\
\href{https://github.com/openai/CLIP/blob/main/data/rendered-sst2.md}{\texttt{RenderedSST2}} & Sentiment from text images & 8,741 & 2 & 448x448 \\
\href{https://archive.ics.uci.edu/dataset/178/semeion+handwritten+digit}{\texttt{SEMEION}} & Handwriting & 1,593 & 10 & 16x16 \\
\href{https://cs.stanford.edu/~acoates/stl10/}{\texttt{STL10}} & General objects & 13,000 & 10 & 96x96 \\
\href{https://vision.princeton.edu/projects/2010/SUN/}{\texttt{SUN397}} & Scenes & 108,753 & 397 & 960x800 \\
\href{http://ufldl.stanford.edu/housenumbers/}{\texttt{SVHN}} & Digits & 630,420 & 10 & 32x32 \\
\href{https://www.kaggle.com/datasets/jessicali9530/stanford-cars-dataset}{\texttt{StanfordCars}} & Cars & 16,185 & 196 & 704x512 \\
\href{https://www.csie.ntu.edu.tw/~cjlin/libsvmtools/datasets/multiclass.html#usps}{\texttt{USPS}} & Handwriting & 9,298 & 10 & 16x16 \\
\bottomrule
\end{tabular}
\end{table}


\newpage

\section*{Code}


\end{document}
